\documentclass[lualatex,ja=standard]{bxjsarticle}
\usepackage[utf8]{inputenc}
\usepackage{amsmath,amssymb,amsfonts}
\usepackage[dvipdfmx]{graphicx}
\usepackage{float}
\usepackage{listings}
\usepackage[dvipdfmx]{xcolor}
\usepackage{url}
\usepackage{here}

% 段落頭を 3–5 スペース相当に
\setlength{\parindent}{2em}
\setlength{\parskip}{0.5\baselineskip}

\begin{document}

\begin{flushleft}
Your Name\\
Your Student Number\\
Course Title\\
Submission Date\\
\end{flushleft}

\vspace{2\baselineskip}

\begin{center}
{\Large\bfseries The Essay Title Is Centered with a Correct Capitalization}
\end{center}

\vspace{2\baselineskip}

Each paragraph must start with an indentation of three to five spaces. It is strongly
recommended to refrain from initiating a new paragraph after only one or two
sentences. The standard font is Times New Roman, 12‑point size. Double‑spacing
is not required in this course, as all documents are processed in the electronic
format. The introduction should consist of a single paragraph. The body should
adhere to the principle of ``one idea per unit.'' Analogous to the introduction, the
conclusion should also be a single paragraph.

\par

The following paragraph must start with an indentation. In‑text citations are
essential when ideas or data from other resources (Armstrong, 2025) are used.
In‑text citation can function as part of a sentence, as Bickerton (2020) demonstrated.
The current format is left‑justified, frequently recommended in academic writing.
However, fully justified writing, with text aligned along both left and right margins,
is acceptable, for it is the default style of many word processors.

\par

The essay must contain at least 300 words. \dots

\vspace{2\baselineskip}

\section*{References}

(Follow the instructions on the textbook/class slides)

\end{document}
